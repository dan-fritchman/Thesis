%
% # Thesis
% ## The main latex entrypoint
%
% Adapted from the template courtesy Paul Vojta
% https://www.overleaf.com/latex/templates/uc-berkeley-thesis-template/mfzmtxfqvtxx
% 

\documentclass{ucbthesis}
\usepackage{biblatex}
\usepackage{rotating} % provides sidewaystable and sidewaysfigure
\usepackage{minted}
\usepackage[fencedCode,citations,definitionLists,hashEnumerators,smartEllipses,pipeTables,tableCaptions,hybrid]{markdown}
\usepackage{svg}

\usepackage{hyperref}
%%begin novalidate
\markdownSetup{rendererPrototypes={
  link = {\href{#2}{#1}},
  image = {\begin{figure}[hbt!]
    \centering
    \includegraphics{#3}%
    \ifx\empty#4\empty\else
    \caption{#4}%
    \fi
    \label{fig:#1}%
    \end{figure}}
}}
%%end novalidate

% If the Grad. Division insists that the first paragraph of a section
% be indented (like the others), then include this line:
% \usepackage{indentfirst}

\addtolength{\abovecaptionskip}{\baselineskip}
\bibliography{refs}
\hyphenation{mar-gin-al-ia}
\hyphenation{bra-va-do}

\begin{document}

% # 
% # Declarations for Front Matter
% # 
\title{An Integrated Circuit Design Framework for Human, Computer, and ML Designers}
\author{Dan Fritchman}
\degreesemester{Fall}
\degreeyear{2023}
\degree{Doctor of Philosophy}
\chair{Professor Vladimir Stojanovic}
\othermembers{Professor Kris Pister \\
  Professor Alper Atamturk}
\numberofmembers{3}
\field{Electrical Engineering \& Computer Sciences}

\maketitle
\copyrightpage

% # 
% # Abstract
% # 
\begin{abstract}
\begin{markdown}
\markdownInput{abstract.md}
\end{markdown}
\end{abstract}

\begin{frontmatter}

% # 
% # Dedication
% # 
\begin{dedication}
\null\vfil
\begin{center}
\begin{markdown}
\markdownInput{dedication.md}
\end{markdown}
\end{center}
\vfil\null
\end{dedication}

% # 
% # Contents
% # 
\tableofcontents
\clearpage
\listoffigures
\clearpage
\listoftables

% # 
% # Acknowledgements
% # 
\begin{acknowledgements}
\begin{markdown}
\markdownInput{acknowledgements.md}
\end{markdown}
\end{acknowledgements}

\end{frontmatter}

\pagestyle{headings}

% # 
% # Content
% # 
\begin{markdown}
\markdownInput{content.md}
\end{markdown}

% # 
% # Bibliography
% # 
\printbibliography

\end{document}
